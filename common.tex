% Microtype
\RequirePackage{microtype}

% Polyglossia: french
\RequirePackage{polyglossia}
\setdefaultlanguage{french}

% Fonts: Fira and JetBrains Mono
\RequirePackage{fontspec}
\RequirePackage{unicode-math}

\setromanfont[BoldFont={Fira Sans},ItalicFont={Fira Sans Light Italic}]{Fira Sans Light}
\setsansfont[BoldFont={Fira Sans}]{Fira Sans Light}
\setmonofont[BoldFont={JetBrains Mono},ItalicFont={JetBrains Mono ExtraLight Italic}]{JetBrains Mono ExtraLight}[Scale=0.9]
\setfontfamily{\titlefont}{Fira Sans ExtraBold}
\setfontfamily{\boxfont}{Fira Sans Bold}
\setfontfamily{\bigtitlefont}{Fira Sans ExtraBold}[Scale=4]

% Luacode IS NOT MANDATORY 
% If you don't require it, you can compile faster with XeLaTeX
% but microtype won't work as nicely as with LuaLaTeX
%\RequirePackage{luacode}

% Colors
\RequirePackage[table]{xcolor}

\definecolor[named]{UGLiPurple}{HTML}{9F648C}
\definecolor[named]{UGLiRed}{HTML}{cc2936}
\definecolor[named]{UGLiOrange}{HTML}{eb8258}
\definecolor[named]{UGLiYellow}{HTML}{CFB04A}
\definecolor[named]{UGLiGreen}{HTML}{8FB349}
\definecolor[named]{UGLiDarkGreen}{HTML}{248935}
\definecolor[named]{UGLiBlue}{HTML}{267487}
\definecolor[named]{UGLiDarkBlue}{HTML}{1F466D}

% General text formatting
\RequirePackage{colortbl}

\newcommand{\alternaterowcolors}[1][UGLiBlue]{
  \rowcolors[]{1}{#1!10}{#1!0}
}% to alternate colors
\RequirePackage{enumerate}
\RequirePackage{numprint}
\newcommand{\np}{\numprint}
\setlength{\parindent}{0pt}
\renewcommand{\baselinestretch}{1.2}

% Boxes: Packages
\RequirePackage[minted,most]{tcolorbox}
\setminted{autogobble=true,breaklines=true}
%\RequirePackage{fontawesome5}

% Boxes: modubox

% This is the first modubox try using fontawesome5 icons
%\newenvironment*{modubox}[3]{\begin{tcolorbox}[colback=#3!10!white,colframe=#3!75!white,title=#1 {~\boxfont #2},left=1ex,right=1ex,boxrule=0.5pt,breakable,arc=2pt]}{\end{tcolorbox}}

\newenvironment{modubox}[2]{
  \begin{tcolorbox}[title={\color{#2}\boxfont #1},
      colback=white, breakable,
      boxrule=0pt,
      colframe=white,
      colbacktitle=white,
      attach boxed title to top left={xshift=-3.4mm,yshift=.2mm},
      frame hidden,
      sharp corners,
      enhanced,
      borderline west={5pt}{1pt}{#2!50},
      colback=white]
    }
    {\end{tcolorbox}}

% Boxes: various code environments
\newenvironment*{pyc}{\begin{modubox}{Python}{UGLiPurple}}{
  \end{modubox}}

\newenvironment*{pys}{\begin{modubox}{Python}{UGLiPurple}}{
  \end{modubox}}

\newenvironment*{html}{\begin{modubox}{HTML}{UGLiGreen}}{
  \end{modubox}}

\newenvironment*{css}{\begin{modubox}{CSS}{UGLiOrange}}{
  \end{modubox}}

\newenvironment*{js}{\begin{modubox}{Javascript}{UGLiBlue}}{
  \end{modubox}}

\newenvironment*{cpp}{\begin{modubox}{C++}{UGLiOrange}}{
  \end{modubox}}

\newenvironment*{clang}{\begin{modubox}{C}{UGLiOrange}}{
  \end{modubox}}

% Boxes: various text environments
\newenvironment*{definition}[1][]{\begin{modubox}{Définition#1}{UGLiGreen}}{\end{modubox}}

\newenvironment*{notation}[1][]{\begin{modubox}{Notation#1}{UGLiBlue}}{\end{modubox}}

\newenvironment*{exemple}[1][]{\begin{modubox}{Exemple#1}{UGLiBlue}}{\end{modubox}}


\newenvironment*{propriete}[1][]{\begin{modubox}{Propriété#1}{UGLiRed}}{\end{modubox}}

\newenvironment*{methode}[1][]{\begin{modubox}{Méthode#1}{UGLiPurple}}{\end{modubox}}

\newenvironment*{remarque}[1][]{\begin{modubox}{Remarque#1}{UGLiBlue}}{\end{modubox}}

\newenvironment*{aretenir}[1][]{\begin{modubox}{À retenir#1}{UGLiGreen}}{\end{modubox}}

% Boxes: exo
\newcounter{exonumber}
\setcounter{exonumber}{0}
\newenvironment*{exercice}[1][]{\stepcounter{exonumber}\begin{modubox}{Exercice \theexonumber #1
    }{UGLiOrange}}{\end{modubox}}

% Compatibility
\newcommand{\pythoninline}[1]{\mintinline{python}{#1}}
\newcommand{\og}{«}
\newcommand{\fg}{»}

% General math
\newcommand\barmin[1]{\overline{#1\vphantom{b}}\,}
\newcommand\barmaj[1]{\overline{#1\vphantom{X}}}
\newcommand\card{\textrm{Card}}


\newenvironment{pmatrice}															% Environnement pmatrice pour des écritures serrées
{\renewcommand\arraystretch{0.7}\begin{pmatrix}}
    {\end{pmatrix}\renewcommand\arraystretch{1}}
\newcommand\repaff{\left(\ O \ ; \ I\ ; \ \ J\ \right)}								% Point et coordonnées dans le plan
\newcommand\rep{\left(O \ ; \ \vv{\imath}, \ \vv{\jmath}\right)}					% Repère du plan
\newcommand\repe{\left(O \ ; \ \vv{\imath}, \ \vv{\jmath}, \ \vv{k} \right)}		% Repère de l'espace

\newcommand\pc[3]{#1\left( #2 \ ; \ #3 \right)}										% Point et coordonnées dans le plan
\newcommand\pcf[1]{\pc{#1}{x_{#1}}{y_{#1}}}											% Point du plan et ses coordonnées dépendant du nom.
\newcommand\pce[4]{\textrm{#1}\left( #2 \ ; \ #3 \ ; \ #4 \right)}					% Idem dans l'espace
\newcommand\pcfe[1]{\pc{#1}{x_{#1}}{y_{#1}}{z_{#1}}}								% Idem


\newcommand\vc[3]{\vv{#1}\begin{pmatrice} #2 \\ #3 \end{pmatrice}}					% Vecteur du plan et coordonnées
\newcommand\vce[4]{\vv{#1}\begin{pmatrice} #2 \\ #3 \\ #4 \end{pmatrice}}			% Vecteur de l'espace et coordonnées
\newcommand\cc[2]{\begin{pmatrice} #1 \\ #2 \end{pmatrice}}							% Coordonnées colonne
\newcommand\cce[3]{\begin{pmatrice} #1 \\ #2 \\ #3 \end{pmatrice}}					% Idem espace
\newcommand\norme[1]{\| \vv{#1} \|}													% Norme
\newcommand\ps[2]{\ensuremath{\vv{#1}\cdot\vv{#2}}}									% Produit scalaire de deux vecteurs du plan
\newcommand\anglo[2]{\left( \vv{#1}  , \ \vv{#2} \right)}							% Angle orienté
\newcommand\N{\ensuremath{\mathbf{N}}} 												% Naturels
\newcommand\Z{\ensuremath{\mathbf{Z}}} 												% Relatifs
\newcommand\D{\ensuremath{\mathbf{D}}} 												% Décimaux
\newcommand\Q{\ensuremath{\mathbf{Q}}}  											% Rationnels
\newcommand\R{\ensuremath{\mathbf{R}}} 												% Réels
\newcommand\C{\ensuremath{\mathbf{C}}}

\newcommand\oio[2]{\left]#1\ ;\ #2\right[}											% Intervalle borné ouvert
\newcommand\oif[2]{\left]#1\ ;\ #2\right]}											% Intervalle borné semi-ouvert semi-fermé
\newcommand\fio[2]{\left[#1\ ;\ #2\right[}											% Intervalle borné semi-fermé semi ouvert
\newcommand\fif[2]{\left[#1\ ;\ #2\right]}											% Intervalle borné fermé
\newcommand\iif[1]{\left]-\infty\ ;\ #1\right]}										% Coupure inférieure fermée
\newcommand\iio[1]{\left]-\infty\ ;\ #1\right[}										% Coupure inférieure ouverte
\newcommand\fii[1]{\left[#1\ ;\ +\infty\right[}										% Coupure supérieure fermée
\newcommand\oii[1]{\left]#1\ ;\ +\infty\right[}										% Coupure supérieure ouverte

\newcommand{\e}{\textrm{e}}															% exponentielle
\newcommand\courbe[1]{\ensuremath{\mathcal{C}_{#1}}}								% C_f

\newcommand\bis{$^{\textrm{\footnotesize bis}}$}									% Bis
\newcommand\eme{$^{\textrm{e}}$\ }                                                  % ème
\newcommand{\er}{$^{\textrm{er}}$\ }                                                % er
\newcommand{\ere}{$^{\textrm{re}}$\ }                                               % ère



\newcounter{questionnumber}
\newcommand{\question}{\addtocounter{questionnumber}{1}\textbf{\thequestionnumber.~}}
\newcommand{\resetquestion}{\setcounter{questionnumber}{0}}






% \division[2]{n}
% \afficheresultat

\def\makeatletter{\catcode`\@11\relax}
\def\makeatother{\catcode`\@12\relax}
\makeatletter

\def\@makeother#1{\catcode`#1=12\relax}

\def\@ifnextchar#1#2#3{%
  \let\reserved@d=#1%
  \def\reserved@a{#2}\def\reserved@b{#3}%
  \futurelet\@let@token\@ifnch}
\def\@ifnch{%
  \ifx\@let@token\@sptoken
    \let\reserved@c\@xifnch
  \else
    \ifx\@let@token\reserved@d
      \let\reserved@c\reserved@a
    \else
      \let\reserved@c\reserved@b
    \fi
  \fi
  \reserved@c}
\begingroup
\def\:{\global\let\@sptoken= } \:  % this makes \@sptoken a space token
\def\:{\@xifnch} \expandafter\gdef\: {\futurelet\@let@token\@ifnch}
\endgroup


\def\@ifstar#1{\@ifnextchar *{\@firstoftwo{#1}}}
\long\def\@dblarg#1{\@ifnextchar[{#1}{\@xdblarg{#1}}}
\long\def\@xdblarg#1#2{#1[{#2}]{#2}}

\long\def \@gobble #1{}

\long\def\@firstofone#1{#1}
\long\def\@firstoftwo#1#2{#1}
\long\def\@secondoftwo#1#2{#2}

\let\@empty\empty

\def\@carcube#1#2#3#4\@nil{#1#2#3}

\def\@star@or@long#1{%
  \@ifstar
  {\let\l@ngrel@x\relax#1}%
  {\let\l@ngrel@x\long#1}}

\let\l@ngrel@x\relax
\def\newcommand{\@star@or@long\new@command}
\def\new@command#1{%
  \@testopt{\@newcommand#1}0}
\def\@newcommand#1[#2]{%
\@ifnextchar [{\@xargdef#1[#2]}%
  {\@argdef#1[#2]}}
\long\def\@argdef#1[#2]#3{%
  \@ifdefinable #1{\@yargdef#1\@ne{#2}{#3}}}
\long\def\@xargdef#1[#2][#3]#4{%
  \@ifdefinable#1{%
    \expandafter\def\expandafter#1\expandafter{%
      \expandafter
      \@protected@testopt
      \expandafter
      #1%
      \csname\string#1\expandafter\endcsname
      {#3}}%
    \expandafter\@yargdef
    \csname\string#1\endcsname
    \tw@
    {#2}%
    {#4}}}

\def\@testopt#1#2{%
  \@ifnextchar[{#1}{#1[#2]}}
\def\@protected@testopt#1{%%
  \ifx\protect\@typeset@protect
    \expandafter\@testopt
  \else
    \@x@protect#1%
  \fi}

\long\def\@yargdef#1#2#3{%
  \@tempcnta#3\relax
  \advance \@tempcnta \@ne
  \let\@hash@\relax
  \edef\reserved@a{\ifx#2\tw@ [\@hash@1]\fi}%
  \@tempcntb #2%
  \@whilenum\@tempcntb <\@tempcnta
  \do{%
    \edef\reserved@a{\reserved@a\@hash@\the\@tempcntb}%
    \advance\@tempcntb \@ne}%
  \let\@hash@##%
  \l@ngrel@x\expandafter\def\expandafter#1\reserved@a}
\long\def\@reargdef#1[#2]#3{%
  \@yargdef#1\@ne{#2}{#3}}

\def\renewcommand{\@star@or@long\renew@command}
\def\renew@command#1{%
  {\escapechar\m@ne\xdef\@gtempa{{\string#1}}}%
  \expandafter\@ifundefined\@gtempa
  {\@latex@error{\string#1 undefined}\@ehc}%
  {}%
  \let\@ifdefinable\@rc@ifdefinable
    \new@command#1}
    \long\def\@ifdefinable #1#2{%
      \edef\reserved@a{\expandafter\@gobble\string #1}%
      \@ifundefined\reserved@a
        {\edef\reserved@b{\expandafter\@carcube \reserved@a xxx\@nil}%
          \ifx \reserved@b\@qend \@notdefinable\else
            \ifx \reserved@a\@qrelax \@notdefinable\else
              #2%
            \fi
          \fi}%
        \@notdefinable}
        \let\@@ifdefinable\@ifdefinable
          \long\def\@rc@ifdefinable#1#2{%
          \let\@ifdefinable\@@ifdefinable
            #2}

            \def\@ifundefined#1{%
              \expandafter\ifx\csname#1\endcsname\relax
                \expandafter\@firstoftwo
              \else
                \expandafter\@secondoftwo
              \fi}

              \newcount\@tempcnta
              \newcount\@tempcntb

              \long\def\@whilenum#1\do #2{\ifnum #1\relax #2\relax\@iwhilenum{#1\relax
                    #2\relax}\fi}
              \long\def\@iwhilenum#1{\ifnum #1\expandafter\@iwhilenum
              \else\expandafter\@gobble\fi{#1}}
              \long\def\@whiledim#1\do #2{\ifdim #1\relax#2\@iwhiledim{#1\relax#2}\fi}
              \long\def\@iwhiledim#1{\ifdim #1\expandafter\@iwhiledim
              \else\expandafter\@gobble\fi{#1}}
              \long\def\@whilesw#1\fi#2{#1#2\@iwhilesw{#1#2}\fi\fi}
        \long\def\@iwhilesw#1\fi{#1\expandafter\@iwhilesw
    \else\@gobbletwo\fi{#1}\fi}
\def\@nnil{\@nil}
\def\@empty{}

\ifx\@@input\@undefined
  \let\@@input\input
\fi

\def\input{\@ifnextchar\bgroup\@iinput\@@input}
\def\@iinput#1{\@@input#1 }
\input{colordvi.tex}
\newbox\nb@box
\newcount\nb@a
\newcount\nb@b
\newcount\iter@
\newcommand\division[2][2]{%
  \def\dividende@{#2}\def\base@{#1}\iter@\@ne\division@{#2}{#1}}
\newcommand\division@[2]{%
  \setbox\nb@box\hbox{\kern0.5em#1\kern0.5em}%
  \nb@a#1 \nb@b#1 \divide\nb@b#2
  \vtop{%
    \begingroup
    \multiply\nb@b#2 \advance\nb@a-\nb@b
    \hbox to\wd\nb@box{\hfil#1\hfil}%
    \vskip3pt\hrule height0pt width\wd\nb@box\vskip3.4pt
    \hbox to\wd\nb@box{\hfil\bf\Red{\number\nb@a}\kern0.5em}%
    \expandafter\xdef\csname reste@\number\iter@\endcsname{\number\nb@a}%
    \endgroup}%
  \setbox\nb@box\hbox{8}\vrule height\ht\nb@box depth3.5ex
  \setbox\nb@box\hbox{\kern0.5em\ifnum#2>\nb@b #2\else\number\nb@b\fi\kern0.5em}%
  \vtop{%
    \hbox to\wd\nb@box{\kern0.5em#2\hfil}%
    \vskip3pt\hrule height0.4pt width\wd\nb@box\vskip3pt
    \hbox{%
      \csname @\ifnum\nb@b>\z@ first\else second\fi oftwo\endcsname
      {\advance\iter@\@ne\gdef\maxiter{\number\iter@}%
        \expandafter\division@\expandafter{\number\nb@b}{#2}}%
      {\kern0.5em\number\nb@b\xdef\maxiter{\number\iter@}}}}}

\newcommand\afficheresultat{$(\dividende@)_{10}=(\afficheresultat@\maxiter)_{\base@}$}
\newcommand\afficheresultat@[1]{%
\csname reste@#1\endcsname
\ifnum#1>\@ne
  \expandafter\expandafter\expandafter\afficheresultat@
\else
  \expandafter\@gobble
\fi{\number\numexpr#1-1}}
\makeatother