\documentclass{draftnsiarticle}
\begin{document}
%\newcommand{\alternaterowcolors}[1][UGLiBlue]{\rowcolors[]{1}{#1!10}{#1!0}}

% \newcommand{\tabulardefault}{
%  \arrayrulecolor{black}
%  \rowcolors{1}{white}{white}
%}
%\newcommand{\tabularstyled}[1][UGLiBlue]{
%  \arrayrulecolor{white}
% \rowcolors[]{1}{#1!10}{#1!0}
%}
%\newcommand{\ths}{\boxfont\bfseries\color{white}\boldmath}

\newcommand{\UGLiTabBackground}{white}
\newcommand{\tabstyle}[1][UGLiOrange]{
    \renewcommand{\UGLiTabBackground}{#1}
    \arrayrulecolor{white}
    \rowcolors[]{1}{#1!10}{#1!0}
}
\newcommand{\ccell}{\cellcolor{\UGLiTabBackground}\boxfont\bfseries\color{white}\boldmath}
\newcommand{\tabdefault}{
    \arrayrulecolor{black}
    \rowcolors{1}{white}{white}
}
\begin{tabular}{|c|c|c|}
    a & z & e \\
    a & z & e \\
    a & z & e \\
    a & z & e \\
\end{tabular}

\tabstyle

\begin{tabular}{|c|c|c|}
    \ccell   a &  z &  e \\
    a           & z        & e        \\
    a           & z        & e        \\
    a           & z        & e        \\
\end{tabular}

\tabdefault

\begin{tabular}{|c|c|c|}
    \ccell   a &  z &  e \\
    a           & z        & e        \\
    a           & z        & e        \\
    a           & z        & e        \\
\end{tabular}

\tabstyle[UGLiGreen]

\begin{tabular}{|c|c|c|}
    \ccell   a &  z &  e \\
    a           & z        & e        \\
    a           & z        & e        \\
    a           & z        & e        \\
\end{tabular}
\end{document}