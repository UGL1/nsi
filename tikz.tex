
\definecolor{quadrillage@color}		{cmyk}	{1,0.2,0.3,0.1}
\definecolor{pointbord@color}{named} {black}
\definecolor{point@color}{named} {UGLiOrange}
\definecolor{pointilles@color}		{rgb}	{.5,.5,.5}


%_____________________________________________________________________________________________________________________________________________________________________
%
\newcommand{\reperevl}[4] %(xmin,ymin,xmax,ymax)
	{
	\draw [thin, quadrillage@color](#1,#2) grid (#3,#4);
	\draw [thick, black] (#1,0) -- (#3,0);
	\draw [thick, black] (0,#2) -- (0,#4);
	\draw [ black](0,0) node[below left] {O};
	\draw [thick, ->,>=latex,black] (0,0) -- (1,0) node[midway,below]  {\footnotesize $\vv{\imath}$};
	\draw [thick, ->,>=latex,black] (0,0) -- (0,1) node[midway,left]  {\footnotesize $\vv{\jmath}$};
	}
%_____________________________________________________________________________________________________________________________________________________________________________
%
\newcommand{\reperev}[4] %(xmin,ymin,xmax,ymax)
	{
	\draw [very thin, color = quadrillage@color!50, step = .5](#1,#2) grid (#3,#4);
	\draw [thin, color = quadrillage@color](#1,#2) grid (#3,#4);
	\draw [thick,color = black] (#1,0) -- (#3,0);
	\draw [thick,color = black] (0,#2) -- (0,#4);
	\draw [color = black](0,0) node[below left] {O};
	\draw [thick, ->,>=latex,color = black] (0,0) -- (1,0) node[midway,below]  {\footnotesize $\vv{\imath}$};
	\draw [thick, ->,>=latex,color = black] (0,0) -- (0,1) node[midway,left]  {\footnotesize $\vv{\jmath}$};
	}
%_____________________________________________________________________________________________________________________________________________________________________________
%
\newcommand{\repereal}[4] %(xmin,ymin,xmax,ymax)
	{
	\draw [thin, quadrillage@color](#1,#2) grid (#3,#4);
	\draw [->,>=latex, thick] (#1,0) -- (#3,0);
	\draw [->,>=latex, thick] (0,#2) -- (0,#4);
	\draw (0,0) node[below left] {O};
	\draw (1,0) node[below] {I} node {\tiny $|$};
	\draw (0,1) node[left] {J} node[rotate =90] {\tiny $|$};
	}
%_____________________________________________________________________________________________________________________________________________________________________________
%
\newcommand{\reperea}[4] %(xmin,ymin,xmax,ymax)
	{
	\draw [very thin, quadrillage@color!50, step = .5](#1,#2) grid (#3,#4);
	\draw [thin, quadrillage@color](#1,#2) grid (#3,#4);
	\draw [->,>=latex, thick] (#1,0) -- (#3,0);
	\draw [->,>=latex, thick] (0,#2) -- (0,#4);
	\draw (0,0) node[below left] {O};
	\draw (1,0) node[below] {I} node {\tiny $|$};
	\draw (0,1) node[left] {J} node[rotate =90] {\tiny $|$};
	}
%_____________________________________________________________________________________________________________________________________________________________________________
%
\newcommand{\reperecompl}[4] %(xmin,ymin,xmax,ymax)
	{
	\draw [thin, quadrillage@color](#1,#2) grid (#3,#4);
	\draw [->,>=latex, thick] (#1,0) -- (#3,0);
	\draw [->,>=latex, thick] (0,#2) -- (0,#4);
	\draw (0,0) node[below left] {$0$};
	\draw (1,0) node[below] {$1$} node {\tiny $|$};
	\draw (0,1) node[left] {$i$} node[rotate =90] {\tiny $|$};
	}
%_____________________________________________________________________________________________________________________________________________________________________________
%
\newcommand{\reperecomp}[4] %(xmin,ymin,xmax,ymax)
	{
	\draw [very thin, quadrillage@color!50, step = .5](#1,#2) grid (#3,#4);
	\draw [thin, quadrillage@color](#1,#2) grid (#3,#4);
	\draw [->,>=latex, thick] (#1,0) -- (#3,0);
	\draw [->,>=latex, thick] (0,#2) -- (0,#4);
	\draw (0,0) node[below left] {0};
	\draw (1,0) node[below] {1} node {\tiny $|$};
	\draw (0,1) node[left] {$i$} node[rotate =90] {\tiny $|$};
	}
%_____________________________________________________________________________________________________________________________________________________________________________
%
\newcommand{\reperec}[6] %(xmin,ymin,xmax,ymax,xlabel,ylabel)
	{
	\draw [very thin, quadrillage@color!50, step = .5](#1,#2) grid (#3,#4);
	\draw [thin, quadrillage@color](#1,#2) grid (#3,#4);
	\draw [->,>=latex, thick] (#1,0) -- (#3,0) ;
	\draw (#3,0) node[below left]{\footnotesize #5};
	\draw [->,>=latex, thick] (0,#2) -- (0,#4);
	\draw (0,#4) node[rotate = 90, above left]{\footnotesize #6};
	}
%_____________________________________________________________________________________________________________________________________________________________________________
%
\newcommand{\reperecl}[6] %(xmin,ymin,xmax,ymax,xlabel,ylabel)
	{

\draw [thin, quadrillage@color](#1,#2) grid (#3,#4);
\draw [->,>=latex, thick] (#1,0) -- (#3,0) ;
\draw (#3,0) node[below left]{\footnotesize #5};
\draw [->,>=latex, thick] (0,#2) -- (0,#4);
\draw (0,#4) node[rotate = 90, above left]{\footnotesize #6};
	}
%_____________________________________________________________________________________________________________________________________________________________________________
%

\newcommand{\ball}{node [circle,fill=pointbord@color, inner sep=.12em]{}node [circle,fill=point@color, inner sep=.08em]{}}
\newcommand{\point}[3] %(x,y,name)
	{
	\draw [thick, pointilles@color, dashed](#1,0) -- (#1,#2) \ball -- (0,#2);
	\ifthenelse{\( \lengthtest{#1 cm > 0 cm } \) \AND \( \lengthtest{#2 cm >0 cm } \)}
		{\draw (#1,#2) node[above right]{#3};}
		{}
	\ifthenelse{\( \lengthtest{#1 cm > 0 cm } \) \AND \( \lengthtest{#2 cm<0 cm} \)}
		{\draw (#1,#2) node[below right]{#3};}
		{}
	\ifthenelse{\( \lengthtest{ #1 cm < 0 cm} \) \AND \( \lengthtest{ #2 cm >0 cm} \)}
		{\draw (#1,#2) node[above left]{#3};}
		{}
	\ifthenelse{\( \lengthtest{ #1 cm < 0 cm} \) \AND \( \lengthtest{ #2 cm <0 cm} \)}
		{\draw (#1,#2) node[below left]{#3};}
		{}
	}
%_____________________________________________________________________________________________________________________________________________________________________________
%
\newcommand{\pointx}[3] %(x,y,name)
	{
	\draw [thick, pointilles@color, dashed](#1,0) -- (#1,#2) -- (0,#2);
	\ifthenelse{\( \lengthtest{#1 cm > 0 cm } \) \AND \( \lengthtest{#2 cm >0 cm } \)}
		{\draw (#1,#2) node[above right]{#3};}
		{}
	\ifthenelse{\( \lengthtest{#1 cm > 0 cm } \) \AND \( \lengthtest{#2 cm<0 cm} \)}
		{\draw (#1,#2) node[below right]{#3};}
		{}
	\ifthenelse{\( \lengthtest{ #1 cm < 0 cm} \) \AND \( \lengthtest{ #2 cm >0 cm} \)}
		{\draw (#1,#2) node[above left]{#3};}
		{}
	\ifthenelse{\( \lengthtest{ #1 cm < 0 cm} \) \AND \( \lengthtest{ #2 cm <0 cm} \)}
		{\draw (#1,#2) node[below left]{#3};}
		{}
	}
%_____________________________________________________________________________________________________________________________________________________________________________
%
\newcommand{\pointc}[5] %(x,y,label1,label2,name)
	{
	\draw [thick, pointilles@color, dashed](#1,0) -- (#1,#2)\ball -- (0,#2);
	\ifthenelse{\( \lengthtest{#1 cm > 0cm}\)\AND\( \lengthtest{#2 cm>0 cm }\)}
		{\draw (#1,#2) node[above right]{#5};
		 \draw (0,#2) node[left]{#4};
		 \draw (#1,0) node[below]{#3};
	}
	{}
	\ifthenelse{\(\lengthtest{#1 cm> 0 cm }\)\AND\( \lengthtest{ #2 cm<0 cm}\)}
		{\draw (#1,#2) node[below right]{#5};
			 \draw (0,#2) node[left]{#4};
		 \draw (#1,0) node[above]{#3};}
		{}
	\ifthenelse{\(\lengthtest{#1cm < 0cm}\)\AND\(\lengthtest{#2cm>0cm}\)}
		{\draw (#1,#2) node[above left]{#5};
			 \draw (0,#2) node[right]{#4};
		 \draw (#1,0) node[below]{#3};}
		{}
	\ifthenelse{\(\lengthtest{#1cm <0cm}\)\AND\(\lengthtest{#2cm<0cm}\)}
		{\draw (#1,#2) node[below left]{#5};
			 \draw (0,#2) node[right]{#4};
		 \draw (#1,0) node[above]{#3};}
		{}
	}
%_____________________________________________________________________________________________________________________________________________________________________________
%
\newcommand{\pointcx}[5] %(x,y,label1,label2,name)
	{
	\draw [thick, pointilles@color, dashed](#1,0) -- (#1,#2) -- (0,#2);

	\ifthenelse{\( \lengthtest{#1 cm > 0cm}\)\AND\( \lengthtest{#2 cm>0 cm }\)}
		{\draw (#1,#2) node[above right]{#5};
		 \draw (0,#2) node[left]{#4};
		 \draw (#1,0) node[below]{#3};
	}
	{}
	\ifthenelse{\(\lengthtest{#1 cm> 0 cm }\)\AND\( \lengthtest{ #2 cm<0 cm}\)}
		{\draw (#1,#2) node[below right]{#5};
			 \draw (0,#2) node[left]{#4};
		 \draw (#1,0) node[above]{#3};}
		{}
	\ifthenelse{\(\lengthtest{#1cm < 0cm}\)\AND\(\lengthtest{#2cm>0cm}\)}
		{\draw (#1,#2) node[above left]{#5};
			 \draw (0,#2) node[right]{#4};
		 \draw (#1,0) node[below]{#3};}
		{}
	\ifthenelse{\(\lengthtest{#1cm <0cm}\)\AND\(\lengthtest{#2cm<0cm}\)}
		{\draw (#1,#2) node[below left]{#5};
			 \draw (0,#2) node[right]{#4};
		 \draw (#1,0) node[above]{#3};}
		{}
	}
%_____________________________________________________________________________________________________________________________________________________________________________
%
\newcommand{\carreaux}[2]
	{
	\carreauxseyes{#1}{#2}
	}
%_____________________________________________________________________________________________________________________________________________________________________________
%
\newcommand{\case}{\tikz[baseline=.2em]{\draw[drop shadow, fill=white] (0,0) rectangle (1em,1em);}}
\newcommand{\valide}[1]{\hfill #1 \case}
%_____________________________________________________________________________________________________________________________________________________________________________
%
\newcommand{\qcmtrois}[4]																	% QCM : question puis trois choix
	{
	#1
	\begin{enumerate}[\hspace*{2em}\bfseries a : \hspace*{-3em}\case\hspace*{3em}]
	\item 	#2
	\item 	#3
	\item	#4\\
\end{enumerate}
	}
%_____________________________________________________________________________________________________________________________________________________________________________
%
\newcommand{\qcmquatre}[5]																	% QCM : question puis quatre choix
	{
	#1
	\begin{enumerate}[\hspace*{2em}\bfseries a : \hspace*{-3em}\case\hspace*{3em}]
	\item 	#2
	\item 	#3
	\item	#4
	\item 	#5\\
\end{enumerate}
	}
%_____________________________________________________________________________________________________________________________________________________________________________
%
\newcommand{\secteur}[3]   % {x,y,angledepart,anglefin}
	{
	\draw (#1)  ++(#2:.2) arc (#2:#3:.2);
	}
%_____________________________________________________________________________________________________________________________________________________________________________
%
\newcommand{\touche}[1] %touche calculette
{\tikz[baseline=.2em]{\draw[drop shadow ,fill=white,rounded corners=2pt] (0,0) rectangle (2em,1em);\draw (1em,.5em) node{\tiny{#1}};}}
%_____________________________________________________________________________________________________________________________________________________________________________
%
\newcommand{\carreauxseyes}[2]
{
	\begin{tikzpicture}
	\draw[fill = white](0,0) rectangle (#1,#2);
	\draw[quadrillage@color!20,thin,ystep=.2,xstep=.8] (0,0) grid (#1,#2);
	\draw[quadrillage@color!70,step=.8,ystep=.8](0,0) grid (#1,#2);
	\end{tikzpicture}
}

\newcommand{\petitscarreaux}[2]
{
    \begin{tikzpicture}
        \draw[fill = white](0,0) rectangle (#1,#2);
        \draw[quadrillage@color!70,step=.5,ystep=.5](0,0) grid (#1,#2);
    \end{tikzpicture}
}

%_____________________________________________________________________________________________________________________________________________________________________________
%
\newcommand{\arbreproba}%arbre type bac tes {A}{B}{C}
{\begin{tikzpicture}[grow = right,level distance=8em,level 1/.style={sibling distance=7em},
level 2/.style={sibling distance=3em}]
\node {}
 child {node {\abdeux} child {node {\abquatre} edge from parent node[below=.5em] {\alsix}} child {node {\abtrois}edge from parent node[above=.5em]
 {\alcinq}}edge from parent node[below=.5em] {\aldeux}}
 child {node {\abun} child {node {\abquatre} edge from parent node[below=.5em] {\alquatre}} child {node {\abtrois}edge from parent node[above=.5em]
 {\altrois}}edge from parent node[above=.5em] {\alun}};
\end{tikzpicture}}

%_____________________________________________________________________________________________________________________________________________________________________________
%
\newcommand{\temps}[1]{\hfill
\begin{tikzpicture}[baseline,yscale = 1.2]
\draw [color = gray!50,fill = gray!50](.7ex,.2ex) arc(0:180:.2ex);
\draw [color = gray!50,fill = gray!50](.7ex,0.8ex) arc(0:-180:.2ex);
\draw[fill=gray!50] (0,0) rectangle (1ex,.2ex);
\draw (.9ex,.2ex) arc(0:180:.4ex);
\draw (.9ex,1ex) arc(0:-180:.4ex);
\draw[color = gray!50] (.5ex,.3ex) -- (.5ex,.8ex);
\draw[fill=gray!50] (0,1ex) rectangle+ (1ex,.2ex);
\end{tikzpicture}\scriptsize{\textbf{\ #1min}}\normalsize}
%_____________________________________________________________________________________________________________________________________________________________________________
%
\newcommand{\ligne}{\tikz{\draw (0,0)--(\linewidth,0);}}