\documentclass{nsibook}
\usepackage{lipsum}

\begin{document}
\begin{titlepage}
    \begin{center}
        \includegraphics[width=12cm]{yin_yang_python.png}\\[2em]

        {\bigtitlefont \LARGE\color{gray} NSI1}

        {\titlefont\Large\color{gray} Spécialité numérique et sciences informatiques en classe de première\\[2em]}

        {\color{gray}\textbf{Lycée Rabelais\\ Saint Brieuc\\ 2023-2024}}
    \end{center}
\end{titlepage}
\chapter{Un titre}
 
\introduction{Commençons par là.}
\begin{aretenir}
    \begin{itemize}
        \item ceci ;
        \item cela ;
        \item cela aussi.
    \end{itemize}
\end{aretenir}

\section{Une section}
\subsection{Une sous-section}
\subsubsection{Une sous-sous-section}
\lipsum[5]
\begin{pyc}
    \begin{minted}{python}
        def f(x: int) -> int:
            return x + 1 # un commentaire
    \end{minted}
\end{pyc}
\lipsum[5]
\begin{figure}
    \begin{center}
        \begin{tabular}{ccc}
            \rowcolor{UGLiOrange!75}
            \color{white}\textbf{premier} & \color{white}\textbf{deuxième} & \color{white}\textbf{troisième} \\
            machin                        & bidule                         & chose                           \\
            zinzin                        & schmurtz                       & truc                            \\
            rebidule                      & rechose                        & remachin                        \\
        \end{tabular}
    \end{center}
    \caption{Une table}
\end{figure}

\begin{exercice}
    Un truc.\\
    \question Faire ci.\\
    \question Faire ça.\\
\end{exercice}
\end{document}
